\documentclass{article}

\usepackage[utf8]{inputenc}
\usepackage[russian]{babel}
\usepackage{amssymb, amsmath, euscript}
\usepackage[left=30mm, right=20mm, top=20mm, bottom=15mm]{geometry}
\usepackage{amsthm}
\usepackage{minted}
\usepackage{multirow, hhline}
\usepackage[table]{xcolor}

\title{Алгоритмы раскраски 3-дольных графов}
\author{Алексей Журавлев}
\date{11 декабря 2015 г.}

\newtheorem*{statement}{Утв}

\begin{document}
\maketitle

\section{Введение}

Задачи поиска приближённого решения различных NP-трудных задач решаются людьми уже несколько десятилетий. 
Примером такой задачи, в которой человечество слабо преуспело за последние 30 лет является задача о правильной раскраске графа в оптимальное число цветов.
В этой статье описаны алгоритмы Виджерсона и Бергера-Ромпеля раскраски 3-раскрашиваемых графов в $O(\sqrt{n})$ и $O(\sqrt{\frac{n}{log(n)}})$ цветов соответственно, где $n$ - число вершин в графе.
Будет доказана корректность работы алгоритмов с точки зрения числа цветов и полиномиальность времени работы.
Также будет приведена конкретная реализация алгоритмов и примеры их запусков на случайных графах.

\section{Немного истории}

Хронология появления полиномиальных алгоритмов раскраски 3-раскрашиваемого графа.

\begin{enumerate}
\item Виджерсон в 1983 году впервые публикует статью, в которой описан алгоритм покраски в $O(\sqrt{n})$ цветов.
\item Бергер, Ромпель, 1990 --- улучшение, раскраска в $O(\sqrt{\frac{n}{log(n)}})$ цветов.
\item Блум, 1992 --- раскраска в $O(n^{\frac{3}{8}})$ цветов.
\item Каргер, Мотвани, Судан, 1994 --- раскраска в $O(n^{\frac{1}{4}})$ цветов.
\item Кламтак, 2007 --- раскраска в $O(n^{0.2072})$ цветов.
\end{enumerate}

Стоит заметить, что нижние оценки для NP-трудности этой задачи всё ещё далеки от этих результатов

\begin{enumerate}
\item Карп в 1972 в своей книге публикует доказательство NP-трудности поиска оптимальной 3-раскраски для 3-дольного графа.
\item Канна, Линиал, Сафра, 1993 --- доказательство NP-трудности поиска раскраски 3-дольного графа в 4 цвета.
\end{enumerate}

\section{Алгоритм Виджерсона}

Пусть дан граф $G=(V, E), |V| = n$,  про который известно, что его можно раскрасить в 3 цвета. Обозначим $N(v) = \{u: (u, v) \in E \} $ - множество соседей вершины $v$ в графе $G$

Алгоритм Виджерсона основан на простой идее, которая состоит в следующем: 
если граф, красится в 3 цвета, то для любой вершины $v$: $N(v)$ можно правильно покрасить в 2 цвета.
Это так, потому что в правильной 3-раскраске у $N(v)$ цвета вершин отличны от цвета вершины $v$ и, следовательно, принимают только 2 значения. Поиск же правильной 2-раскраски делается за полиномиальное время тривиальным поиском в глубину.

Другая идея состоит в том, что граф, в котором степень каждой вершины не превосходит $d$, можно быстро и правильно покрасить в $d + 1$ цвет.

Это действительно так: можно просто просматривать вершины по очереди, и красить каждую вершину в минимально возможный цвет c учётом уже покрашенных. Т. к. степень каждой вершины не превосходит $d$, то номер выбранного цвета не может быть больше, чем $d + 1$.

Тогда алгоритм будет состоять в следующем:
\begin{enumerate}
\item Найдём вершину $v$ степени $\ge \sqrt{n}$. Рассмотрим множество её соседей $N(v)$. Если такой вершины $v$ не нашлось, переходим к шагу $(3)$.
\item Покрасим $N(v)$ правильно в 2 цвета и удалим из графа вместе со смежными рёбрами. Данные 2 цвета больше не будем использовать при дальнейшей покраске других вершин. Вернёмся на шаг $(1)$.
\item Последовательно просмотрим все не покрашенные вершины и покрасим их в минимально возможный цвет с учётом уже покрашенных.
\end{enumerate}

Раскраска графа правильная по построению алгоритма. Докажем оценку на число цветов, в которые будет покрашен граф.
\begin{statement}
Алгоритм Виджерсона красит 3-раскрашиваемый граф в $O(\sqrt{n})$ цветов.
\end{statement}

$\blacktriangleleft$
Заметим, что шаг (2) не может быть выполнен больше чем $\sqrt{n}$ раз, т.к. каждый раз, он удаляет из графа хотя бы $\sqrt{n}$, вершин, а суммарно в графе $n$ вершин.
Каждый раз шаг (2) использует 2 новых цвета. Итого на шаге (2) будет использовано суммарно $\le 2\sqrt{n}$ цветов.
На шаге (3) не может быть использовано больше $\sqrt{n} + 1$ цветов, т.к. степень каждой вершины на шаге (3) строго меньше $\sqrt{n}$.
Итого суммарно использовано $\le \sqrt{n} + 1 + 2\sqrt{n} = 3\sqrt{n} + 1 = O(\sqrt{n})$ цветов.
$\blacksquare$

\bigskip

Покажем полиномиальность алгоритма.

\begin{statement}
Алгоритм Виджерсона строит покраску 3-раскрашиваемого графа за время $O(n^{2})$
\end{statement}
$\blacktriangleleft$
При покраске множества вершин в 2 цвета каждый раз запускается поиск в глубину. При этом, если вершина покрашена в этом поиске в глубину, то вершина и все инцидентные рёбра больше не будут посещены никаким поиском в глубину, т. к. исключаются из рассмотрения после покраски. Значит суммарное время работы всех поисков в глубину на шаге (2): $O(|V| + |E|) = O(n^2)$. На шаге (1) можно найти вершину максимальной степени за $O(n)$, поэтому суммарное время работы шага (1): $O(n^{\frac{3}{2}})$. Выбор цвета на шаге (3) выполняется за $O(\sqrt{n})$, вершин на шаге (3) нужно обработать не больше $O(n)$. Суммарное время на шаге (3) не больше, чем $O(n^{\frac{3}{2}})$. Итоговое время работы $O(n^2)$
$\blacksquare$

\section{Реализация алгоритма Виджерсона}

Приведём реализацию алгоритма Виджерсона на языке Python. Версия интепретатора 3.5.0. Граф договоримся хранить в виде списков смежности. Вершины имеют номера от 0 до $n-1$.


Для начала опишем основные вспомогательные функции для реализации алгоритма. 


Первая из них $color\_subset(graph, important\_verteces, colors, min\_color)$ --- принимает на вход граф, список вершин, которые нужно покрасить в 2 цвета, список соответствующий текущей покраске вершин ($None$ - соответствует не покрашенной вершине), а также номер последнего не использованного цвета. Покраска производится тривиально: поиском в глубину. Для экономии памяти на стеке, функция поиска в глубину внесена внутрь основной как замыкание. Возвращает функция значение $min\_color + 2$, как новый минимальный не использованный цвет. Если в процессе обхода выяснилось, что окрестность не является 2-раскрашиваемой, то возвращается $None$.

Вторая функция $recalculate\_degrees(graph, important\_verteces, degrees)$ --- пересчитывает степени всех смежных вершин после выполнения предыдущей процедуры, в соответствии с тем, что покрашенные вершины считаются удалёнными из графа. На вход подаётся граф, тот же список уже покрашенных вершин и список текущих степеней.

Третья функция $calculate\_greedy\_rest(graph, colors)$ --- красит множество вершин графа, которые ещё не покрашены, жадным алгоритмом (в минимально возможный цвет).
 
\begin{minted}
[
frame=lines,
framesep=2mm,
baselinestretch=1.2,
bgcolor=lightgray,
fontsize=\footnotesize,
linenos,
label=help$\_$functions.py
]
{python}

def color_subset(graph, important_verteces, colors, min_color):
    vertices_set = set(important_verteces)
    def dfs(v, shift):
        colors[v] = min_color + shift
        for incident in graph[v]:
            if incident in vertices_set:
                if colors[incident] is None:
                    if not dfs(incident, 1 - shift):
                        return False
                elif colors[incident] == colors[v]:
                    return False
        return True
    for vertex in important_verteces:
        if colors[vertex] is None:
            if not dfs(vertex, 0):
                return None
    return min_color + 2


def recalculate_degrees(graph, important_verteces, degrees):
    for vertex in important_verteces:
        for incident in graph[vertex]:
            degrees[incident] -= 1
    for vertex in important_verteces:
        degrees[vertex] = 0
    return
    
   
def calculate_greedy_rest(graph, colors):
    n = len(graph)
    for vertex in range(n):
        if colors[vertex] is None:
            neighbour_colors = set(range(n))
            for neighbour in graph[vertex]:
                if colors[neighbour] is not None and colors[neighbour] in neighbour_colors:
                    neighbour_colors.remove(colors[neighbour])
            colors[vertex] = min(neighbour_colors)   

\end{minted}

\bigskip

Теперь приведём реализацию основного алгоритма. Алгоритм оформлен в виде класса, в конструктор которому подаётся граф, после чего там же происходит раскраска графа алгоритмом Виджерсона. Реализация в точности повторяет словесное описание из предыдущего пункта статьи: берётся вершина максимальной степени, рассматривается множество её соседей и красится в 2 цвета, после чего пересчитываются степени всех вершин. Когда максимальная степень вершины меньше, чем $\sqrt{n}$, переходим к жадной раскраске оставшейся части графа.

Получить раскраску графа пользователь может методом класса $get\_coloring()$.

\bigskip

\begin{minted}
[
frame=lines,
framesep=2mm,
baselinestretch=1.2,
bgcolor=lightgray,
fontsize=\footnotesize,
linenos,
label=widgerson.py
]
{python}

from math import sqrt
from help_functions import color_subset, recalculate_degrees, calculate_greedy_rest


class Widgerson:

    def __init__(self, graph):
        n = len(graph)
        degrees = [len(incident_list) for incident_list in graph]
        border = sqrt(n)
        self.colors = [None] * n
        min_color = 0
        while True:
            max_degree = max(degrees)
            if max_degree < border:
                break
            current_vertex = degrees.index(max_degree)
            min_color = color_subset(graph, graph[current_vertex], self.colors, min_color)
            recalculate_degrees(graph, set(graph[current_vertex]), degrees)
        calculate_greedy_rest(graph, self.colors)


    def get_coloring(self):
        return self.colors

\end{minted}

\section{Алгоритм Бергера-Ромпеля}

Является во всех смыслах улучшением предыдущего алгоритма, обобщая при этом всё ту же общую идею.

В то время, как Виджерсон пользуется тем, что окрестность любой вершины является двудольной, Бергер и Ромпель пользуются тем, что окрестность любого независимого множества в графе $G$, которое одноцветно в некоторой правильной 3-раскраске, является двудольной из тех же рассуждений. Вопрос лишь в том, как такое множество искать.

Для $ S \subset V $ будем обозначать $N(S) = \bigcup\limits_{u \in S}N(u)$.

Алгоритм будет состоять в следующем:

\begin{enumerate}
\item Положим $ S = \emptyset $
\item Если существует вершина $u \in V$, которая добавляет хотя бы $\sqrt{\frac{n}{\log{n}}}$ соседних вершин к соседним вершинам множества $S$, то есть формально $|N(u) \setminus N(S)| \ge \sqrt{\frac{n}{\log{n}}}$, то добавим $u$ в множество $S$. Если такой вершины нет, переходим к шагу 5.
\item Если $|S| \ge 3\log{n}$, переходим к шагу 4, иначе возвращаемся на шаг 2.
\item Для каждого $ C \subset S: |C| = \log{n} $, являющегося независимым множеством, запустим алгоритм покраски в 2 цвета множества $N(C)$, пока одно из них не покрасится правильно в 2 цвета. После этого удаляем покрашенные вершины из графа и не используем эти 2 цвета при дальнейшей покраске. Возвращаемся к шагу 1.
\item Для всех вершин, которые ещё остались в множестве S, рассмотрим последовательно их окрестности и раскрасим эти окрестности в 2 цвета, удаляя их после этого из графа.
\item Все оставшиеся вершины просматриваем последовательно и красим жадно: в минимально возможный цвет с учётом текущей раскраски.
\end{enumerate}

Только пункт 4 этого алгоритма ставит под вопрос его корректность. Покажем, что он правомерен.

\begin{statement}
В пункте 4 алгоритма найдётся независимое множество $C$, такое что окрестность $N(C)$ можно правильно раскрасить в 2 цвета.
\end{statement}
$\blacktriangleleft$
Заметим, что $|S| \ge 3\log{n}$. При этом граф $G$ --- 3-раскрашиваемый, а значит из принципа Дирихле в $S$ существует подмножество $C$, такое что $|C| \ge \log{n}$ и в правильной раскраске покрашено в один цвет. Тогда $N(C)$ будет в этой раскраске покрашено в 2 цвета.
$\blacksquare$

\bigskip

Итак, алгоритм строит правильную раскраску. Докажем оценку на число цветов в ней.

\begin{statement}
Алгоритм Бергера-Ромпеля красит 3-раскрашиваемый граф в $O(\sqrt{\frac{n}{\log{n}}})$ цветов
\end{statement}
$\blacktriangleleft$
Покраска вершин происходит на 4, 5, 6 шагах алгоритма. 

Заметим, что на 4 шаге алгоритма, после покраски из графа удаляется $\ge \log{n}\sqrt{\frac{n}{\log{n}}}=\sqrt{n\log{n}}$ вершин, т.к. по построению множества на шаге 2 каждая вершина добавляет $\sqrt{\frac{n}{\log{n}}}$ своих соседей, а $|C| \ge |\log{n}|$. Так как всего в графе $G$ $n$ вершин, то шаг 4 не может быть выполнен больше, чем $\sqrt{\frac{n}{\log{n}}}$ раз. Каждый раз используется 2 цвета, значит суммарно на шаге 4 используется $\le 2\sqrt{\frac{n}{\log{n}}}$ цветов.

На шаге 5 нужно последовательно покрасить в 2 цвета окрестности менее чем $3\log{n}$ вершин. Для этого требуется менее чем $6\log{n}$ цветов.

На шаге 6 у каждой вершины степень меньше, чем $\sqrt{\frac{n}{\log{n}}}$, т.к. в противном случае вершина была бы добавлена в множество $S$. Значит жадный алгоритм покрасит все оставшиеся вершины в $\le\sqrt{\frac{n}{\log{n}}} + 1$ цветов.

Итого, общее число цветов $\le 2\sqrt{\frac{n}{\log{n}}} + 6\log{n} + \sqrt{\frac{n}{\log{n}}} + 1 = O(\sqrt{\frac{n}{\log{n}}})$
$\blacksquare$

Остаётся показать полиномиальность времени работы алгоритма.

\begin{statement}
Алгоритм Бергера-Ромпеля красит 3-раскрашиваемый граф за время $O(n^6)$.
\end{statement}
$\blacktriangleleft$
Посчитаем число операций на шаге 4. Шаг 4 выполняется $\le \sqrt{\frac{n}{\log{n}}}$ раз. Каждый раз запускается серия из $(3\log{n})^{\log{n}} \le n^3$ поисков в глубину, каждый из которых который работает $O(n^2)$. Итого $O(n^6)$ операций на 4 шаге.

На шаге 2 нужно для каждой вершины проверить, как её добавление влияет на множество $S$. Это можно тривиально сделать за $O(n^2)$. Суммарно шаг 2 выполняется $\le 3\log{n}\sqrt{\frac{n}{\log{n}}}$ раз. Итого $O(n^3)$ операций на этом шаге.

На шаге 5 просматривается не более, чем $3\log{n}$ окрестностей, каждая из которых красится за время $O(n^2)$. Итого $O(n^3)$ операций на этом шаге.

На шаге 6 линейным проходом просматриваются все вершины и каждый раз выбирается наименьший цвет. $O(n^2)$ операций на этом шаге.

Итак, получили общее время работы $O(n^6)$.
$\blacksquare$

\bigskip

Заметим, что оценка получилась достаточно грубая, и в реальности можно добиться лучшей, но для доказательства полиномиальности этого достаточно.

\section{Реализация алгоритма Бергера-Ромпеля}

Опишем для начала ещё одну вспомогательную функцию.

$get\_all\_neighbours(graph, important\_vertices, colors)$ --- возвращает множество всех соседей для данного множества вершин, при этом только те вершины, которые ещё не покрашены (не удалены с точки зрения алгоритма).

\begin{minted}
[
frame=lines,
framesep=2mm,
baselinestretch=1.2,
bgcolor=lightgray,
fontsize=\footnotesize,
linenos,
label=help$\_$functions.py
]
{python}

def get_all_neighbours(graph, important_vertices, colors):
    neighbours = set()
    for vertex in important_vertices:
        for incident in graph[vertex]:
            if colors[incident] is None and incident not in important_vertices:
                neighbours.add(incident)
    return neighbours

\end{minted}

Теперь можно описать основной алгоритм, который оформлен в виде класса с интерфейсом, аналогичным интерфейсу для алгоритма Виджерсона.

\begin{minted}
[
frame=lines,
framesep=2mm,
baselinestretch=1.2,
bgcolor=lightgray,
fontsize=\footnotesize,
linenos,
label=berger$\_$rompel.py
]
{python}

from math import log2, sqrt
from itertools import combinations
from help_functions import get_all_neighbours, color_subset, calculate_greedy_rest


class BergerRompel:

    def __init__(self, graph):
        n = len(graph)
        self.colors = [None] * n
        logn = round(log2(n) + 0.5)
        border = round(sqrt(n / logn) + 0.5)
        min_color = 0
        while True:
            current_size = 0
            current_s = set()
            current_neighbours = set()
            for vertex in range(n):
                count = 0
                local_set = set()
                for incident in graph[vertex]:
                    if self.colors[incident] is None and incident not in current_neighbours\
                            and incident not in current_s:
                        local_set.add(incident)
                        count += 1
                if count >= border:
                    current_neighbours.update(local_set)
                    current_s.add(vertex)
                    current_size += 1
                if current_size >= 3 * logn:
                    for combination in combinations(current_s, logn):
                        neighbours = get_all_neighbours(graph, combination, self.colors)
                        result = color_subset(graph, neighbours, self.colors, min_color)
                        if result is None:
                            for neighbour in neighbours:
                                self.colors[neighbour] = None
                        else:
                            min_color = result
                            break
                    break
            if current_size < 3 * logn:
                for vertex in current_s:
                    neighbours = get_all_neighbours(graph, [vertex], self.colors)
                    if len(neighbours) > 0:
                        min_color = color_subset(graph, neighbours, self.colors, min_color)
                break
        calculate_greedy_rest(graph, self.colors)

    def get_coloring(self):
        return self.colors

\end{minted}

\bigskip

Реализация дословно повторяет описание алгоритма.

\section{Тестовые запуски алгоритмов}

Для проверки правильности работы реализованных алгоритмов нужно научиться проверять правильность возвращаемой раскраски, а так же генерировать случайные 3-раскрашиваемые графы.

Определим функцию проверки правильности раскраски.

\begin{minted}
[
frame=lines,
framesep=2mm,
baselinestretch=1.2,
bgcolor=lightgray,
fontsize=\footnotesize,
linenos,
label=tests.py
]
{python}

def check_coloring_correct(graph, coloring):
    n = len(graph)
    flag = True
    for vertex in range(n):
        for incident in graph[vertex]:
            if coloring[vertex] == coloring[incident]:
                flag = False
    return flag

\end{minted}

\bigskip

Также определим функцию генерации случайного 3-дольного графа с размерами долей $n, m, l$, и вероятностью появления ребра $p$.

\begin{minted}
[
frame=lines,
framesep=2mm,
baselinestretch=1.2,
bgcolor=lightgray,
fontsize=\footnotesize,
linenos,
label=tests.py
]
{python}

import random

def bernoulli(p):
    return random.uniform(0, 1) <= p


def generate_random_graph(n, m, l, p):
    total = n + m + l
    graph = []
    for i in range(total):
        graph.append([])
    for i in range(n):
        for j in range(m):
            if bernoulli(p):
                graph[i].append(n + j)
                graph[n + j].append(i)
        for k in range(l):
            if bernoulli(p):
                graph[i].append(n + m + k)
                graph[n + m + k].append(i)
    for j in range(m):
        for k in range(l):
            if bernoulli(p):
                graph[n + j].append(n + m + k)
                graph[n + m + k].append(n + j)
    return graph

\end{minted}

\bigskip

Тестирование проведём на маленьких графах: треугольник, 2 несвязанных ребра. Получим корректные раскраски в обоих случаях.

Далее запустим оба алгоритма, на максимально допустимом с точки зрения времени числе вершин при различных значениях $p$.
Получим следующие результаты.

\begin{tabular}{|c|c|c|c|c|c|}
\hline
\multicolumn{3}{|c}{Виджерсон} & \multicolumn{3}{|c|}{Бергер-Ромпель}
\\
\hline
n & p & число цветов & n & p & число цветов
\\ \hline
1000 & 0.5 & 6 & 1000 & 0.5 & 21 \\ \hline
1000 & 0.05 & 28 & 1000 & 0.05 & 62 \\ \hline
1000 & 0.2 & 12 & 1000 & 0.1 & 53 \\ \hline
1500 & 0.1 & 22 & - & - & - \\ \hline
2000 & 0.05 & 28 & - & - & - \\ \hline
3000 & 0.05 & 33 & - & - & - \\
\hline

\end{tabular}

\bigskip

Заметим, что число цветов не превышает заявленных оценок, однако на реальных практических примерах, на которых реализации удаётся запустить, алгоритм Виджерсона работает эффективнее алгоритма Бергера-Ромпеля.

Наиболее эффективно алгоритм Виджерсона работает на графах с большой плотностью рёбер.

Алгоритм Бергера-Ромпеля в данных примерах работает хуже из-за $6\log{n}$ в оценке на число цветов, которое в асимптотике уходит в $O(\sqrt{\frac{n}{\log{n}}})$, однако при $n=1000$ несёт значительный вклад.

Тем самым алгоритм Виджерсона показывает себя достаточно быстрым, простым и эффективным на практике, по сравнению с алгоритмом Бергера-Ромпеля.

\section{Итоги}

В рамках этого обзора удалось рассмотреть только простейшие алгоритмы раскраски 3-дольных графов из множества алгоритмов, рассмотренных в начале этой статьи. Тем не менее, можно отметить, что простой и эффективный с точки зрения времени алгоритм Виджерсона на реальных примерах работает лучше, чем более сложные алгоритмы с асимптотически-лучшими оценками. Однако асимптотическую теорию нельзя не брать в расчёт, учитывая тенденции увеличения вычислительных мощностей с каждым годом.

\begin{thebibliography}{1}

\bibitem{1}
  Bonnie Berger, John Rompel,
  A Better Perfomance Guarantee for Approximate Graph Coloring,
  Algoritmica, 1990
  
\bibitem{2}
  Widgerson A.,
  Improving the perfomance guarantee of approximate graph coloring,
  J. ACM, 30(4): 729-735, 1983

\bibitem{3}
  Ryan O'Donnell,
  Coloring 3-Colorable Graphs using SDP,
  CMU Lecture, Spring 2008  

\end{thebibliography}


\end{document}